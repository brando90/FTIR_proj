\documentclass[10pt,a4paper]{article}
%\usepackage[utf8]{inputenc}
\usepackage[margin=1.0in]{geometry} 
\usepackage{amsmath}
\usepackage{esint}
\usepackage{amsfonts}
\usepackage{amssymb}
\usepackage{url}
\usepackage{graphicx}
\usepackage[LGRgreek]{mathastext}
\usepackage{bm}
\usepackage{authblk}

\title{Notes on reconstruction of optical spectra from 64-bit dFT spectrometer}
\author[1]{Derek M. Kita}
\author[2]{Brando Miranda}
\affil[1]{Department of Materials Science \& Engineering, Massachusetts Institute of Technology}
\affil[2]{Center for Brains, Minds, and Machines, Massachusetts Institute of Technology}

\date{\today}
\begin{document}
\maketitle

\section*{Problem Statement \& Objective}
Given the measured interferogram $y$ ($N\times 1$) and a calibration matrix $A$ ($N\times D$), accurately reconstruct the input optical signal $x$ that obeys:
\begin{equation}
y = Ax \label{eq:1}
\end{equation}
where $D\gg N$, and in our case $D=801$, $N=64$.  For our application, there are two types of signals of interest: (1) laser lines that produce sparse spectra, and (2) light sources (like an EDFA) with a broad spectrum (not-sparse).

\section*{L1 and L2 minimization}
Since the problem we are solving is underdefined, there are an infinite number of solutions $x$ that solve Eq.~\ref{eq:1}. However, with prior knowledge of the size of the correct solution's L1- and L2-norm, we can obtain better estimates of $x$ by minimizing:
\begin{equation}
\min_x \Big\{ ||y-Ax||^2 + \alpha_1 ||x||_1 + \alpha_2 ||x||_2^2 \Big\}
\end{equation}
Solving the above corresponds to the ``elastic net'' regularized regression method (removing only the L1 norm corresponds to ``ridge regression'' and removing only the L2 norm corresponds to ``LASSO'').

\section*{Radial Basis Function (RBF) Network}
For some application, such as broad input sources, our prior information can consist of how ``smooth'' the spectrum is.  In this case, we can construct a function $h$ that represents our spectrum and we seek to minimize the following for appropriate $\alpha$, and $c_k$.
\begin{equation}
\min_x \Big\{ ||y-A h(\lambda)||^2 + \alpha \sum_k c_k \int \frac{\partial^k h(\lambda)}{\partial \lambda^k}d\lambda \Big\}
\end{equation}

The solution to the above minimization problem (that puts constraints on the solution's smoothness), is a RBF Network, which approximates $h$ with radial basis functions:
\begin{equation}
h_c(\lambda) = \sum_{d=1}^D c_d e^{-\beta |\lambda - \lambda_d|^2}
\end{equation}
We can then use an appropriate algorithm like stochastic gradient descent (SGD) to solve the following, simpler minimization problem:
\begin{equation}
\min_c \Big\{ ||y-Ah_c(\lambda)||^2\Big\}
\end{equation}

\section*{non-negative Elastic-D smoothing Regularization}
For some application, such as broad input sources, our prior information can consist of how ``smooth'' the spectrum is. To induce such smoothness one can use the forward finite difference matrix $D$ to define the following regularizer $\| Dx \|^2_2 $. The matrix $D$ is defined as follows:

\[
   D=
  \left[ {\begin{array}{ccccccc}
   1 & 0 & 0 &  \cdots & \cdots & \cdots & 0\\
   0 & -1 & 1 &  \cdots & \cdots & \cdots & 0\\
   0 & 0 & -1 &  \cdots & \cdots & \cdots & 0\\
   \vdots & \vdots & \vdots &  \ddots & \dots & \cdots & \vdots\\
   0 & 0 & 0 &  \cdots & -1 & 1 & 0\\
   0 & 0 & 0 &  \cdots & 0 & -1 & 1\\
   0 & 0 & 0 &  \cdots & 0 & 0 & 1\\
  \end{array} } \right]
\]

note that there are other matrices like the central difference matrix with derivatives approximated by $[-1 , 2 ,2]$ but we did not try it. Note that is easy to generalize the difference in the wavelengths are not equidistant by simply taking into the real difference between the input signal. With this we want to solve the regularization problem:

\begin{equation}
\min_{x,x>0} \Big\{ ||y-Ax||^2 + \alpha_1 ||x||_1 + \alpha_2 ||x||_2^2 + \alpha_3 \| D x\|^2_2 \Big\}
\end{equation}

by using the standard trick that $\| Mx\|^2_2 = x^\top M x$ we can write the above as a non-negative quadratic program by using the fact that $\| D x\|^2_2 = x^\top D x$ and $||x||_2^2 = x^\top I x$ where $I$ is the $D$ by $D$ identity. We also use the fact that $x>0$ thus $\| x \|_1 = \sum^D_{d=1} x_i = \boldsymbol{1}^\top x$ to write the above into the following equivalent non-negative quadratic program:

\begin{equation}
\min_{x,x>0} \Big\{ x^{\top}\left( A^\top A + \alpha_2 I + \alpha_3 D^\top D \right)x + (\alpha_1 \boldsymbol{1} - A^\top y)^\top x \Big\}
\end{equation}

this is easy to show using basic linear algebra and matrix rules. Then we simply plug this in to a standard quadratic program solver available in python.

\end{document}